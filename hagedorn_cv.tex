%% Copyright 2006-2015 Xavier Danaux (xdanaux@gmail.com).
%
% This work may be distributed and/or modified under the
% conditions of the LaTeX Project Public License version 1.3c,
% available at http://www.latex-project.org/lppl/.
\documentclass[11pt,a4paper,sans]{moderncv}        % possible options include font size ('10pt', '11pt' and '12pt'), paper size ('a4paper', 'letterpaper', 'a5paper', 'legalpaper', 'executivepaper' and 'landscape') and font family ('sans' and 'roman')

\usepackage[utf8]{inputenc}

% moderncv themes
\moderncvstyle{classic}                             % style options are 'casual' (default), 'classic', 'banking', 'oldstyle' and 'fancy'
\moderncvcolor{blue}                               % color options 'black', 'blue' (default), 'burgundy', 'green', 'grey', 'orange', 'purple' and 'red'
%\renewcommand{\familydefault}{\sfdefault}         % to set the default font; use '\sfdefault' for the default sans serif font, '\rmdefault' for the default roman one, or any tex font name
%\nopagenumbers{}                                  % uncomment to suppress automatic page numbering for CVs longer than one page

\moderncvtheme[blue]{classic}
\usepackage[utf8]{inputenc}
% adjust the page margins
\usepackage[scale=0.8]{geometry}
\AtBeginDocument{\recomputelengths}

% personal data
\firstname{Bastian}
\familyname{Hagedorn}
\address{Einsteinstraße 62}{48149 Münster, Germany}
\phone[fixed]{+49~(0)~251~83-32744}
\email{b.hagedorn@wwu.de}

\usepackage[style=numeric-verb,
            sorting=none, % keep order as in the bib file ...
            giveninits=true,
            defernumbers,
            backend=biber,
maxbibnames=50]{biblatex}

\defbibenvironment{bibliography}
  {\list%
     {\printfield{year}\hspace{1em}\printtext[labelnumberwidth]{\printfield{labelprefix}\printfield{labelnumber}}}
     {\setlength{\topsep}{0pt}% layout parameters from moderncvstyleclassic.sty
      \setlength{\labelwidth}{\hintscolumnwidth}%
      \setlength{\labelsep}{\separatorcolumnwidth}%
      \leftmargin\labelwidth%
      \advance\leftmargin\labelsep%
      }%
      \sloppy\clubpenalty4000\widowpenalty4000}
  {\endlist}
  {\item}


\DeclareNameAlias{default}{first-last}

\usepackage{xstring}
\usepackage{xpatch}
\newbibmacro*{name:bold}[2]{%
  \def\do##1{\iffieldequalstr{hash}{##1}{\bfseries\listbreak}{}}%
  \dolistloop{\boldnames}%
}

\newcommand*{\boldnames}{}

\xpretobibmacro{name:family}{\begingroup\usebibmacro{name:bold}{#1}{#2}}{}{}
\xpretobibmacro{name:given-family}{\begingroup\usebibmacro{name:bold}{#1}{#2}}{}{}
\xpretobibmacro{name:family-given}{\begingroup\usebibmacro{name:bold}{#1}{#2}}{}{}
\xpretobibmacro{name:delim}{\begingroup\normalfont}{}{}

\xapptobibmacro{name:family}{\endgroup}{}{}
\xapptobibmacro{name:given-family}{\endgroup}{}{}
\xapptobibmacro{name:family-given}{\endgroup}{}{}
\xapptobibmacro{name:delim}{\endgroup}{}{}

% Got hashes from the bbl file
\renewcommand*{\boldnames}{}
\forcsvlist{\listadd\boldnames}
  {{f3b57123791f63857d478ac84c802258},
   {7dd9b93410e4a44439b25ea5e2c49d4a}}

% Only print a year once
\newcounter{currentYear}
\DeclareFieldFormat{year}{%
\ifthenelse{\equal{#1}{\arabic{currentYear}}}%
    {}
%{\setcounter{currentYear}{#1}{\bfseries #1}}}
{\setcounter{currentYear}{#1}{#1}}}

\bibliography{publications}

%----------------------------------------------------------------------------------
%            content
%----------------------------------------------------------------------------------
\begin{document}
\nocite{*}
%-----       resume       ---------------------------------------------------------
\makecvtitle

%----------------------------------------------------------------------------------
%            university education
%----------------------------------------------------------------------------------
\section{University Education}
	\cventry{since 2016}
					{Ph.D. studies}
					{University of Münster}
					{Münster, Germany}{}
					{Supervisor: Prof. Sergei Gorlatch \\
					 %Main research interests: High-level programming abstractions for high-performance
					 %computing applications, Programming of modern multi- and many-core processors
                     %My PhD research focuses on performance-portable stencil code generation using Lift by tackling the growing problem of achieving high efficiency when programming the ever changing hardware of today and tomorrow.
                     My PhD research focuses on high-performance GPU code generation using Lift by tackling the growing problem of achieving high efficiency when programming the ever changing hardware of today and tomorrow.
                     I'm one of the main developers of the Lift project which has recently emerged as a promising compilation framework to achieve performance portability.
                     Lift defines a small set of reusable parallel primitives that DSL writers can easily build upon.
                     Lift’s key novelty is its ability to automatically explore the optimization space by using as a system of extensible rewrite rules which encode specific optimizations.
					}

	\cventry{2014 -- 2016}
					{Master of Science in computer science}
					{University of Münster}{Münster, Germany}
					{\textit{Final grade in computer science: excellent with distinction (90\%)}}
					{Thesis title: An Extension of a Functional Intermediate Language for Parallelizing
					 Stencil Computations and its Optimizing GPU Implementation Using OpenCL.\\
					 \textit{Grade for thesis: excellent}
					}

	\cventry{2011 -- 2014}
					{Bachelor of Science in computer science}
					{University of Münster}{Münster, Germany}
					{\textit{Final grade in computer science: very good (81\%)}}
					{Thesis title: Implementation of a Multicast Module for the Floodlight SDN-Controller \\
					 %In this thesis, I extended the Floodlight network controller with a module which enables
					 %a novel approach to multicast communication in software-defined networks.
					 \textit{Grade for thesis: excellent}
					}

%----------------------------------------------------------------------------------
%            research visits
%----------------------------------------------------------------------------------
\section{Research Visits}
    \cventry{07/2018 \\-- 09/2018}
					{Deep Learning Compiler Engineer Intern (3 months)}
					{NVIDIA}
					{Redmond, WA, USA}{}
					{
                    During this internship, I was working on an embedded DSL, IR and compiler for optimizing deep learning applications.
                    I focused especially on CUDA code generation for highly optimized matrix multiplication algorithms.
                    My work on this project simplified managing the deep compute and memory hierarchy on modern GPUs and enabled using program synthesis for design space exploration.
                    }

    \cventry{02/2018 \\-- 04/2018}
					{Visiting researcher (2 months)}
					{University of Glasgow}
					{Glasgow, UK}{}
					{Funded by HPC-Europa3\\
                     During this visit, I investigated the implementation of performance portable HPC applications with Lift including the automatic fusion of multiple compute kernels using rewrite rules.
                     I focused on geometric multigrid methods and the Irish Centre for High-End Computing (ICHEC) supported my visit by providing access to their GPU hardware.
					}

	\cventry{07/2017 \\-- 09/2017}
					{Visiting researcher (2 months)}
					{University of Edinburgh}
					{Edinburgh, UK}{}
					{Funded by HiPEAC\\
                     During this visit, I combined modern auto-tuning techniques with the current Lift code generator.
                     I also evaluated Lift's functional compilation approach compared to state-of-the-art polyhedral compilation.
                     A paper describing the results of this and our previous collaborations has won the \textit{best paper award} at the prestigious International Symposium on Code Generation and Optimization (CGO)~\cite{cgo2018}
					}

	\cventry{02/2017 \\-- 03/2017}
					{Visiting researcher (2 months)}
					{University of Edinburgh}
					{Edinburgh, UK}{}
					{Funded by EuroLab-4-HPC\\
					 During this visit, I extended the Lift compiler, developed at the
					 University of Edinburgh, to enable automatic exploration of stencil-specific optimizations.
					}

	\cventry{04/2016 \\-- 05/2016}
					{Visiting researcher (2 months)}
					{University of Edinburgh}
					{Edinburgh, UK}{}
					{Funded by EuroLab-4-HPC\\
					 During this visit, I extended the Lift compiler to enable the generation of
					 high-performance stencil code for GPUs.
					}

	\cventry{09/2015}
					{Visiting researcher (3 weeks)}
					{HUST University}
					{Wuhan, China}{}
					{Funded by the EC’s 7th Framework Programme MONICA for accelerating the transfer
					 and deployment of research knowledge between European countries and China.
					 During this visit, I implemented an experimental setup for SDN-based multicast,
					 and prepared a research paper on this topic~\cite{humernbrum}
					}

%----------------------------------------------------------------------------------
%            publications
%----------------------------------------------------------------------------------
\printbibheading[title={Publications}]
\printbibliography[heading=none]
% Publications from a BibTeX file without multibib
%  for numerical labels: \renewcommand{\bibliographyitemlabel}{\@biblabel{\arabic{enumiv}}}% CONSIDER MERGING WITH PREAMBLE PART
%  to redefine the heading string ("Publications"): \renewcommand{\refname}{Articles}
%\bibliographystyle{plain}
%\bibliography{publications}



%----------------------------------------------------------------------------------
%            presentations
%----------------------------------------------------------------------------------
\section{Presentations}
    \cventry{04/2018}
					{\normalfont{Talk: \textit{High Performance Stencil Code Generation with Lift}}}{}{}{}
                    {Workshop on Compilers for Parallel Computing (CPC), Dublin, Ireland}
    \cventry{04/2018}
					{\normalfont{Invited Talk: \textit{High Performance Stencil Code Generation with Lift}}}{}{}{}
                    {Dependable Systems Group, Heriot-Watt University Edinburgh, UK}
    \cventry{04/2018}
					{\normalfont{Tutorial: \textit{Lift: Performance Portable Parallel Code Generation via Rewrite Rules}}}{}{}{}
                    {International Symposium on Performance Analysis of Systems and Software (ISPASS), Belfast, UK}
    \cventry{03/2018}
					{\normalfont{Talk: \textit{High Performance Stencil Code Generation with Lift}}}{}{}{}
                    {Scottish Programming Language Seminar (SPLS), University of Glasgow, UK}
    \cventry{02/2018}
					{\normalfont{Talk: \textit{High Performance Stencil Code Generation with Lift}}}{}{}{}
                    {International Symposium on Code Generation and Optimization (CGO), Vienna, Austria}
	\cventry{02/2018}
					{\normalfont{Invited Talk: \textit{High Performance Stencil Code Generation with Lift}}}{}{}{}
					{Research Group on Compiler and Architecture Design, University of Edinburgh, UK}
    \cventry{03/2017}
					{\normalfont{Invited Talk: \textit{Performance Portable Stencil Code Generation with Lift}}}{}{}{}
					{Research Group on Compiler and Architecture Design, University of Edinburgh, UK}


%----------------------------------------------------------------------------------
%            research projects
%----------------------------------------------------------------------------------
\section{Research Projects}
    \cventry{}
                    {\normalfont{\small I have been actively contributing to the following research projects}}{}{}{}{}


	\cventry{since 04/2016}
					{Lift}
					{\textit{A Novel Approach to Achieving Performance Portability on Accelerators}}{}{}
					{Ongoing research, \textit{www.lift-project.org}\\
					 I am one of the main contributors focusing on implementing stencil computations in Lift.
					 I extended the functional Lift IR and enabled the generation
					 of efficient OpenCL kernels for stencil-based applications.
					 The Lift project is a novel approach to generate high-performance OpenCL kernels
					 from high-level functional programs.\\
					}

	\cventry{04/2015}
					{PACXX}
					{\textit{Programming Accelerators with C++}}{}{}{
					 Ongoing research \\
					 I developed an LLVM analysis pass for the PACXX compiler and ported HPC applications
					 to the PACXX programming model resulting in a publication~\cite{haidl}.
					 PACXX is a unified HPC programming model for programming accelerators
					 (GPUs etc.) using pure C++ by implementing a custom compiler
					 (based on the LLVM framework) and a runtime system.
					}

	\cventry{10/2013 \\- 09/2014}
					{OFERTIE EU Project}
					{\textit{OpenFlow Experiment in Real-Time Internet Edutainment}}{}{}
					{
					 I configured the SDN
					 testbed at the University of Münster, conducted several SDN-based experiments
					 and extended the monitoring interface of the Real-Time Framework (RTF)
					 The OFERTIE project aims to use SDN approaches to improve delivery of Real-Time Online
					 Interactive Applications (ROIA).
					}

%----------------------------------------------------------------------------------
%            academic events
%----------------------------------------------------------------------------------
\section{Attended Academic Events}
\cvitem{2018}{CPC - \textit{20th Workshop on Compilers for Parallel Computing}, Dublin, Ireland}
\cvitem{}{ISPASS - \textit{International Symposium on Performance Analysis of Systems and Software}, Belfast, UK}
\cvitem{}{SPLS - \textit{Scottish Programming Languages Seminar}, Glasgow, UK}
\cvitem{}{CGO - \textit{International Symposium on Code Generation and Optimization}, Vienna, Austria}
\cvitem{2017}{Compiler and Programming Language Summit (organized by Google), Munich, Germany}
\cvitem{}{ACASES Summer School (organized by HiPEAC) - \textit{Thirteenth International Summer School on Advanced Computer Architecture and Compilation for High-Performance and Embedded Systems}, Fiuggi, Italy}
\cvitem{}{PUMPS Summer School - \textit{Eighth edition of the Programming and Tuning Massively Parallel Systems summer school}, Barcelona, Spain}
\cvitem{}{SPLS - \textit{Scottish Programming Languages Seminar}, St. Andrews, UK}
\cvitem{2016}{HLPP conference - \textit{9th International Symposium on High-Level Parallel Programming and Applications}, Münster, Germany}
\cvitem{}{UKMAC - \textit{UK Many-Core Developer Conference}, Edinburgh, UK}
\cvitem{}{WadlerFest/LFCS30 - \textit{30th Aniversery of the Laboratory for Foundations of Computer Science}, Edinburgh, UK}
\cvitem{2015}{PRACE Course - \textit{Advanced Parallel Programming with MPI and OpenMP}, Jülich, Germany}
\cvitem{}{PRACE Course - \textit{Node-Level Performance Engineering}, Stuttgart, Germany}

%----------------------------------------------------------------------------------
%            reviewer
%----------------------------------------------------------------------------------
\section{Reviewer}

    \cvitem{2018}{CGO 2018 artifact evaluation commitee}
    \cvitem{}{LCTES 2018 artifact evaluation commitee}
    %\cvitem{2017}{Dr Hans Riegel Fachpreise commitee - (Undergrad Competition)}
	\cvitem{2016 -- 2018}
				 {I have been active as an external reviewer for the following conferences and journals:
                  \emph{Principles and Practice of Parallel Programming (PPoPP),
                    the International Journal of Parallel Programming (IJPP)},
					the \emph{Journal of Supercomputing},
					the journal \emph{Concurrency and Computation: Practice and Experience},
                    the Journal of Applied Geophysics (APPGEO),
					the Parallel Computing Technologies (PaCT),
					the Parallel Computing Conference (ParCo),
					the UKRCON
					and the PSI.
				 }

%----------------------------------------------------------------------------------
%            memberships
%----------------------------------------------------------------------------------
%\section{Memberships}
%    \cvitem{}{ACM Student Member}
%    \cvitem{}{Berufungskomission W2 Praktische Informatik}

%----------------------------------------------------------------------------------
%            teaching
%----------------------------------------------------------------------------------
\section{Teaching}
	\cvitem{Winter 2019}{Teaching assistant for the course: \textit{Operating systems}}
	\cvitem{Winter 2019}{Teaching assistant for the course: \textit{Introduction to programming with Java and Haskell}}
	\cvitem{Summer 2018}{Course design and Lecturer: \textit{Introduction to programming with C and C++}}
	\cvitem{Summer 2018}{Teaching assistant for the course: \textit{Parallel Programming: Multi-Core and GPU}}
	\cvitem{Winter 2017}{Teaching assistant for the course: \textit{Operating systems}}
	\cvitem{Winter 2017}{Teaching assistant for the course: \textit{Introduction to programming with Java and Racket}}
	\cvitem{Summer 2017}{Course design and Lecturer: \textit{Introduction to programming with C and C++}}
	\cvitem{Summer 2017}{Supervised a student project: \textit{Automatic program optimization for modern many-core systems}}
	\cvitem{Winter 2016}{Teaching assistant for the course: \textit{Operating systems}}
	\cvitem{Winter 2015}{Student assistant for the course: \textit{Operating systems}}
	\cvitem{Summer 2015}{Student assistant for the course: \textit{Computer architectures}}
	\cvitem{Winter 2014}{Student assistant for the course: \textit{Operating systems}}

%----------------------------------------------------------------------------------
%            supervised students
%----------------------------------------------------------------------------------
\section{Supervised Undergraduate and Master Students}
    \cvitem{since 02/2018}{Johannes Lenfers (Master): \emph{Implementing Compiler Auto-Tuning Strategies for Design Space Exploration of Lift Programs}}
    \cvitem{02/2018}{Bastian Köpcke (Master): \emph{Efficient GPU Code Generation for FFT Computations in Lift}}
    \cvitem{01/2018}{Martin Lücke (Master): \emph{Efficient Implementation and Optimization of Geometric Multi-Grid Operations in the Lift Framework}}
    \cvitem{03/2018}{Clemens Hesse-Edenfeld (Undergraduate): \emph{Integrating Performance Models for Stencil Computations in Lift}}
    \cvitem{03/2018}{Alexander Dirk Holthaus (Master): \emph{Development of an Analytical Tool for Visualizing and Optimizing Memory Accesses in GPU Kernels}}
    \cvitem{03/2018}{Maurice Heine (Undergraduate): \emph{Implementation of a Visualization Tool for Lift Programs}}


%----------------------------------------------------------------------------------
%            technical skills
%----------------------------------------------------------------------------------
%\section{Technical Skills}
%	\cventry{Programming Languages}
%					{Scala, C/C++, Java}{}{}{}
%					{Experiences:
%					 Stencil support for Lift compiler (Scala),
%				 	 Multicast Module for the Floodlight SDN Controller (Java),
%					 Measurement library for OpenCL (C++),
%					 Implementation of the WiPo architecture (Java),
%					 Monitoring interface extension of RTF (C++)
%					}
%
%	\cventry{Parallel Programming}
%					{OpenCL, CUDA, OpenMP}{}{}{}
%					{Experiences:
%					 Performance portability evaluation of OpenCL Kernels on Intel Xeon (Phi) and NVIDIA Tesla.
%					 JIT compilation of a DSL using LLVM and CUDA Driver API
%					}
%
%	\cventry{Compiler Tools}
%					{LLVM}{}{}{}
%					{Experiences:
%					 Analysis Pass for the PACXX Compiler,
%					 Compiler frontend for self-defined DSL for data parallel applications based on algorithmic skeletons
%					}

\end{document}
