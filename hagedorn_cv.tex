%% Copyright 2006-2015 Xavier Danaux (xdanaux@gmail.com).
%
% This work may be distributed and/or modified under the
% conditions of the LaTeX Project Public License version 1.3c,
% available at http://www.latex-project.org/lppl/.
\documentclass[11pt,a4paper,sans]{moderncv}        % possible options include font size ('10pt', '11pt' and '12pt'), paper size ('a4paper', 'letterpaper', 'a5paper', 'legalpaper', 'executivepaper' and 'landscape') and font family ('sans' and 'roman')

\usepackage[utf8]{inputenc}

% moderncv themes
\moderncvstyle{classic}                             % style options are 'casual' (default), 'classic', 'banking', 'oldstyle' and 'fancy'
\moderncvcolor{blue}                               % color options 'black', 'blue' (default), 'burgundy', 'green', 'grey', 'orange', 'purple' and 'red'
%\renewcommand{\familydefault}{\sfdefault}         % to set the default font; use '\sfdefault' for the default sans serif font, '\rmdefault' for the default roman one, or any tex font name
%\nopagenumbers{}                                  % uncomment to suppress automatic page numbering for CVs longer than one page

\moderncvtheme[blue]{classic}
\usepackage[utf8]{inputenc}
% adjust the page margins
\usepackage[scale=0.8]{geometry}
\AtBeginDocument{\recomputelengths}

% personal data
\firstname{Bastian}
\familyname{Hagedorn}
\address{Einsteinstraße 62}{48149 Münster, Germany}
\phone[fixed]{+49~(0)~251~83-32744}
\email{b.hagedorn@wwu.de}

\usepackage[style=numeric-verb,
            sorting=none, % keep order as in the bib file ...
            giveninits=true,
            defernumbers,
            backend=biber,
maxbibnames=50]{biblatex}

\defbibenvironment{bibliography}
  {\list%
     {\printfield{year}\hspace{1em}\printtext[labelnumberwidth]{\printfield{labelprefix}\printfield{labelnumber}}}
     {\setlength{\topsep}{0pt}% layout parameters from moderncvstyleclassic.sty
      \setlength{\labelwidth}{\hintscolumnwidth}%
      \setlength{\labelsep}{\separatorcolumnwidth}%
      \leftmargin\labelwidth%
      \advance\leftmargin\labelsep%
      }%
      \sloppy\clubpenalty4000\widowpenalty4000}
  {\endlist}
  {\item}


\DeclareNameAlias{default}{first-last}

\usepackage{xstring}
\usepackage{xpatch}
\newbibmacro*{name:bold}[2]{%
  \def\do##1{\iffieldequalstr{hash}{##1}{\bfseries\listbreak}{}}%
  \dolistloop{\boldnames}%
}

\newcommand*{\boldnames}{}

\xpretobibmacro{name:family}{\begingroup\usebibmacro{name:bold}{#1}{#2}}{}{}
\xpretobibmacro{name:given-family}{\begingroup\usebibmacro{name:bold}{#1}{#2}}{}{}
\xpretobibmacro{name:family-given}{\begingroup\usebibmacro{name:bold}{#1}{#2}}{}{}
\xpretobibmacro{name:delim}{\begingroup\normalfont}{}{}

\xapptobibmacro{name:family}{\endgroup}{}{}
\xapptobibmacro{name:given-family}{\endgroup}{}{}
\xapptobibmacro{name:family-given}{\endgroup}{}{}
\xapptobibmacro{name:delim}{\endgroup}{}{}

% Got hashes from the bbl file
\renewcommand*{\boldnames}{}
\forcsvlist{\listadd\boldnames}
  {{f3b57123791f63857d478ac84c802258},
   {7dd9b93410e4a44439b25ea5e2c49d4a}}

% Only print a year once
\newcounter{currentYear}
\DeclareFieldFormat{year}{%
\ifthenelse{\equal{#1}{\arabic{currentYear}}}%
    {}
%{\setcounter{currentYear}{#1}{\bfseries #1}}}
{\setcounter{currentYear}{#1}{#1}}}

\bibliography{publications}

%----------------------------------------------------------------------------------
%            content
%----------------------------------------------------------------------------------
\begin{document}
\nocite{*}
%-----       resume       ---------------------------------------------------------
\makecvtitle

%----------------------------------------------------------------------------------
%            university education
%----------------------------------------------------------------------------------
\section{University Education}
	\cventry{since 2016}
					{Ph.D. studies}
					{University of Münster}
					{Münster, Germany}{}
					{Supervisor: Prof. Sergei Gorlatch \\
					 Main research interests: High-level programming abstractions for high-performance
					 computing applications, Programming of modern multi- and many-core processors
					}

	\cventry{2014 -- 2016}
					{Master of Science in computer science}
					{University of Münster}{Münster, Germany}
					{\textit{Final grade in computer science: excellent with distinction (90\%)}}
					{Thesis title: An Extension of a Functional Intermediate Language for Parallelizing
					 Stencil Computations and its Optimizing GPU Implementation Using OpenCL.\\
					 In this thesis, I extended the LIFT compiler to enable the generation of
					 high-performance stencil code for GPUs from a high-level functional program
					 \textit{Grade for thesis: excellent}
					}

	\cventry{2011 -- 2014}
					{Bachelor of Science in computer science}
					{University of Münster}{Münster, Germany}
					{\textit{Final grade in computer science: very good (81\%)}}
					{Thesis title: Implementation of a Multicast Module for the Floodlight SDN-Controller \\
					 In this thesis, I extended the Floodlight network controller with a module which enables
					 a novel approach to multicast communication in software-defined networks.
					 \textit{Grade for thesis: excellent}
					}

%----------------------------------------------------------------------------------
%            research visits
%----------------------------------------------------------------------------------
\section{Research Visits}
	\cventry{07/2017 \\-- 09/2017}
					{Visiting researcher (2 months)}
					{University of Edinburgh}
					{Edinburgh, UK}{}
					{Funded by HiPEAC\\
                     During this visit I combined modern auto-tuning techniques with the current LIFT code generator.
                     I also evaluated LIFTs functional compilation approach compared to state-of-the-art polyhedral compilation.
					 A paper describing our results is currently under review at the prestigious CGO conference~\cite{cgo2018}
					}

	\cventry{02/2017 \\-- 03/2017}
					{Visiting researcher (2 months)}
					{University of Edinburgh}
					{Edinburgh, UK}{}
					{Funded by the EuroLab-4-HPC\\
					 During this visit, I extended the LIFT compiler, developed at the
					 University of Edinburgh, to enable automatic exploration of stencil-specific optimizations.
					}

	\cventry{04/2016 \\-- 05/2016}
					{Visiting researcher (2 months)}
					{University of Edinburgh}
					{Edinburgh, UK}{}
					{Funded by the EuroLab-4-HPC\\
					 During this visit, I extended the LIFT compiler to enable the generation of
					 high-performance stencil code for GPUs.
					}

%	\cventry{12/2015}
%					{PRACE course participant}
%					{Jülich Supercomputing Centre}
%					{Jülich, Germany}{}
%					{Course: Advanced Parallel Programming with MPI and OpenMP \\
%					 Content: Nonblocking Communication, Virtual Topologies, OpenMP-4.0 Extensions,
%					 Parallel programming models on hybrid systems
%					}

	\cventry{09/2015}
					{Visiting researcher (3 weeks)}
					{HUST University}
					{Wuhan, China}{}
					{Funded by the EC’s 7th Framework Programme MONICA for accelerating the transfer
					 and deployment of research knowledge between European countries and China.
					 During this visit, I implemented an experimental setup for SDN-based multicast,
					 and prepared a research paper on this topic~\cite{humernbrum}
					}

%	\cventry{07/2015}
%					{PRACE course participant}
%					{High Performance Computing Centre Suttgart}
%					{Stuttgart, Germany}{}
%					{Course: Node-Level Performance Engineering \\
%					 Content: Parallel hardware architectures, Roofline performance model, benchmarking
%					 and profiling tools for high-performance systems and applications
%					}

%----------------------------------------------------------------------------------
%            presentations
%----------------------------------------------------------------------------------
\section{Presentations}
	\cventry{03/2017}
					{\normalfont{Invited Talk: \textit{Performance Portable Stencil Code Generation with LIFT}}}{}{}{}
					{Research Group on Compiler and Architecture Design, University of Edinburgh, UK}

\newpage

%----------------------------------------------------------------------------------
%            publications
%----------------------------------------------------------------------------------
\printbibheading[title={Publications}]
\printbibliography[heading=none]
% Publications from a BibTeX file without multibib
%  for numerical labels: \renewcommand{\bibliographyitemlabel}{\@biblabel{\arabic{enumiv}}}% CONSIDER MERGING WITH PREAMBLE PART
%  to redefine the heading string ("Publications"): \renewcommand{\refname}{Articles}
%\bibliographystyle{plain}
%\bibliography{publications}


%----------------------------------------------------------------------------------
%            research projects
%----------------------------------------------------------------------------------
\section{Research Projects}
	\cventry{since 04/2016}
					{LIFT}
					{\textit{A Novel Approach to Achieving Performance Portability on Accelerators}}{}{}
					{Ongoing research, \textit{www.lift-project.org}\\
					 I am one of the main contributors focusing on implementing stencil computations in Lift.
					 I extended the functional LIFT IR and enabled the generation
					 of efficient OpenCL kernels for stencil-based applications.
					 The Lift project is a novel approach to generate high-performance OpenCL kernels
					 from high-level functional programs.\\
					}

	\cventry{04/2015}
					{PACXX}
					{\textit{Programming Accelerators with C++}}{}{}{
					 Ongoing research \\
					 I developed an LLVM analysis pass for the PACXX compiler and ported HPC applications
					 to the PACXX programming model resulting in a publication~\cite{haidl}.
					 PACXX is a unified HPC programming model for programming accelerators
					 (GPUs etc.) using pure C++ by implementing a custom compiler
					 (based on the LLVM framework) and a runtime system.
					}

	\cventry{10/2013 - 09/2014}
					{OFERTIE EU Project}
					{\textit{OpenFlow Experiment in Real-Time Internet Edutainment}}{}{}
					{
					 I configured the SDN
					 testbed at the University of Münster, conducted several SDN-based experiments
					 and extended the monitoring interface of the Real-Time Framework (RTF)
					 The OFERTIE project aims to use SDN approaches to improve delivery of Real-Time Online
					 Interactive Applications (ROIA).
					}

%----------------------------------------------------------------------------------
%            academic events
%----------------------------------------------------------------------------------
\section{Attended Academic Events}
\cvitem{2017}{ACASES (HiPEAC) - \textit{Thirteenth International Summer School on Advanced Computer Architecture and Compilation for High-Performance and Embedded Systems}}
\cvitem{}{PUMPS Summer School - \textit{Programming and Tuning Massively Parallel Systems}}
\cvitem{}{SPLS - \textit{Scottish Programming Languages Seminar}, Edinburgh, UK}
\cvitem{2016}{HLPP - \textit{9th International Symposium on High-Level Parallel Programming and Applications}, Münster, Germany}
\cvitem{}{UKMAC - \textit{UK Many-Core Developer Conference}, Edinburgh, UK}
\cvitem{}{WadlerFest/LCFS30 - \textit{30th Aniversery of the Laboratory for Foundations of Computer Science}, Edinburgh, UK}
\cvitem{2015}{PRACE Course - \textit{Advanced Parallel Programming with MPI and OpenMP}, Jülich, Germany}
\cvitem{}{PRACE Course - \textit{Node-Level Performance Engineering}, Stuttgart, Germany}

%----------------------------------------------------------------------------------
%            reviewer
%----------------------------------------------------------------------------------
\section{Reviewer}
	\cvitem{2016 -- 2017}
				 {I have been active as an external reviewer for the following conferences and journals:
				  \emph{International Journal of Parallel Programming (IJPP)},
					the \emph{Journal of Supercomputing},
					the journal \emph{Concurrency and Computation: Practice and Experience},
					the Parallel Computing Technologies (PaCT),
					the Parallel Computing Conference (ParCo),
					the UKRCON
					and the PSI.
				 }

%----------------------------------------------------------------------------------
%            teaching
%----------------------------------------------------------------------------------
\section{Teaching}
	\cvitem{Winter 2017}{Teaching assistant for the course: \textit{Operating systems}}
	\cvitem{Winter 2017}{Teaching assistant for the course: \textit{Introduction to programming with Java and Racket}}
	\cvitem{Summer 2017}{Course design and Lecturer: \textit{Introduction to programming with C and C++}}
	\cvitem{Summer 2017}{Supervised a student project: \textit{Automatic program optimization for modern many-core systems}}
	\cvitem{Winter 2016}{Teaching assistant for the course: \textit{Operating systems}}
	\cvitem{Winter 2015}{Student assistant for the course: \textit{Operating systems}}
	\cvitem{Summer 2015}{Student assistant for the course: \textit{Computer architectures}}
	\cvitem{Winter 2014}{Student assistant for the course: \textit{Operating systems}}

%----------------------------------------------------------------------------------
%            technical skills
%----------------------------------------------------------------------------------
\section{Technical Skills}
	\cventry{Programming Languages}
					{Scala, C/C++, Java}{}{}{}
					{Experiences:
					 Stencil support for Lift compiler (Scala),
				 	 Multicast Module for the Floodlight SDN Controller (Java),
					 Measurement library for OpenCL (C++),
					 Implementation of the WiPo architecture (Java),
					 Monitoring interface extension of RTF (C++)
					}

	\cventry{Parallel Programming}
					{OpenCL, CUDA, OpenMP}{}{}{}
					{Experiences:
					 Performance portability evaluation of OpenCL Kernels on Intel Xeon (Phi) and NVIDIA Tesla.
					 JIT compilation of a DSL using LLVM and CUDA Driver API
					}

	\cventry{Compiler Tools}
					{LLVM}{}{}{}
					{Experiences:
					 Analysis Pass for the PACXX Compiler,
					 Compiler frontend for self-defined DSL for data parallel applications based on algorithmic skeletons
					}

\end{document}
